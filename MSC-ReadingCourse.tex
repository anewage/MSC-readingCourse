\documentclass[
	11pt,
	aspectratio=169,
]{beamer}

% Setting the colors
\hypersetup{
	colorlinks,
	citecolor=cyan,
	linkcolor=.,
	urlcolor=cyan
}
\usepackage[utf8]{inputenc}
\usepackage[T1]{fontenc}

% Remove these 2 lines
\usepackage[english]{babel}
\usepackage{blindtext}

% Defining a color
\usepackage{xcolor}
\definecolor{uwo-purple}{HTML}{4F2683}
\definecolor{uwo-gray}{HTML}{807F83}

% Refs
\usepackage{csquotes}
\usepackage[backend=bibtex,style=alphabetic]{biblatex}
\addbibresource{refs.bib}

% Phonetic characters
\usepackage{tipa}
\usepackage{tipx}



\usetheme{CambridgeUS}

% Customizing the template
\setbeamercolor{frametitle}{fg=uwo-purple}
\setbeamercolor{title}{fg=uwo-purple}
\setbeamercolor{palette primary}{fg=black, bg=uwo-purple!30!white}
\setbeamercolor{palette secondary}{fg=black, bg=uwo-purple!20!white}
\setbeamercolor{palette tertiary}{bg=uwo-purple}


\begin{document}
	\author{Amir HaghighatiMaleki}
	\title{Design/Cybernetics: Two Sides of One Coin}
	\subtitle{Reading Course Presentation}
	\logo{\includegraphics[width=0.7cm]{resources/uwo-purple.png}}
	\institute{Insight Lab}
	\date{\today}
	\subject{Reading Course Presentation}
	%\setbeamercovered{transparent}
	%\setbeamertemplate{navigation symbols}{}
	\begin{frame}
		\maketitle
		\centering\tiny\hyperlink{http://insight.uwo.ca}{insight.uwo.ca}\\
		\centering\tiny\hyperlink{mailto:ahaghig3@uwo.ca}{ahaghig3@uwo.ca}
	\end{frame}

    \begin{frame}
    	\textit{\rm``The progress of knowledge is at the same time the progress of ignorance.''}\\
    	--- Edgar Morin \cite{morin_1992} \\
    	\vspace{0.4cm}
    	\textit{\rm``The very act of focusing prevents us from seeing.''}\\
    	--- ?? \\
    	\vspace{0.4cm}
    	\textit{\rm``We do not see what we do not see, and what we do not see seems nonexistent.''}\\
    	--- Humberto R. Maturana and Francisco J. Varela \cite{maturana_varela_1987}\\
    	\vspace{0.4cm}
    	\textit{\rm``The only way to overcome second-order deficiencies is with therapies of second order.''}\\
    	--- Heinz von Foerster \cite{vonFoerster_2003}\\
    	\vspace{0.4cm}
    	\begin{figure}
    		\centering
    		\href{https://anewage.github.io}{
    			\includegraphics[width=2cm]{resources/anewage_github_io.png}
    		}
    	\end{figure}
    \end{frame}

    \section*{Prolouge}
        \begin{frame}
        	\frametitle{Prologue I}
        	\begin{columns}
        		\column{0.3\textwidth}
        			\begin{figure}
        				\includegraphics[height=0.7\textheight]{resources/man.png}
        			\end{figure}
        		\column{0.3\textwidth}
        			\begin{figure}
        				\includegraphics[width=\textwidth]{resources/liar.jpg}
        			\end{figure}
        		\column{0.3\textwidth}
        			\begin{figure}
        				\includegraphics[width=\textwidth]{resources/chickenegg.jpg}
        			\end{figure}
        	\end{columns}
        \end{frame}

        \begin{frame}
        \frametitle{Prologue II}
        	{\textbf{Gödel's incompleteness} theorems can be summarized as: \\
        		It is impossible to construct a theoretical language that will not \textcolor{teal}{close on itself}, with the result that there are statements that are \textcolor{teal}{undecidable} in that language.} \\
        	\vspace{0.5cm}
        	{Simply put: There are things that you \textcolor{teal}{cannot explain within a paricular language}.}
        \end{frame}

    \section{What?}
        \subsection{Initial Notes}
            \begin{frame}{Design}
                \framesubtitle{Basics}
                The word \textcolor{teal}{design} has its roots in drawing and designation. \\
        	    \textbf{As a noun} $\longrightarrow$ From Italian \textit{\rm designò}: ``to draw''\\
        	    \textbf{As a verb} $\longrightarrow$ From Latin \textit{\rm designare}: ``to designate, to outline, to appoint''\\
        	    \centering dē- (from, out) + signō (``to mark, to sign''). 
            \end{frame}
            \begin{frame}{Design}
                \framesubtitle{Wide Range of Approaches}
        		\begin{itemize}
        			\item<1->Design as a complex but essentially \textbf{mechanical} action (e.g., some approaches in Software Engineering, Herbert Simon):\\
        			    Assuming a set of specified criteria: design is generating a set of  alternatives and assessing them against the criteria
        			\item<2-> Horst Rittle (1973) $\longrightarrow$ design is essentially for \textcolor{teal}{``wicked problems''}:
        			\textbf{no definitive formulation}, \textbf{no stopping rule}, \textbf{no assessment criteria}
        			\item<3->Latest attempts: \textcolor{teal}{examine the heart of the design activity} instead of \textcolor{red}{prescription} and \textcolor{red}{proscription}: design and cognition (e.g., most interdisciplinary attempts)
        		\end{itemize}
            \end{frame}
            \begin{frame}{Cybernetics}
                \framesubtitle{Basics}
        			\textbf{Cybernetics} (\textbackslash \textipa{saI\texttt{}b@r"netIks}\textbackslash):
        			\begin{itemize}
        				\item<1-> From Greek `kybernetes': `\textcolor{teal}{steersman}'; `kybernan': `\textcolor{teal}{to steer or pilot a ship, direct as a pilot}'
        				\item<2-> Having a goal and taking action to achieve that goal
        				\item<3-> The art of steering \cite{pangaro_web}:
        			\end{itemize}
        			\centering\includegraphics<3>[width=7.2cm]{./resources/steering1.png}
        			\centering\includegraphics<4>[width=7.2cm]{./resources/steering2.png}
        			\centering\includegraphics<5>[width=7.2cm]{./resources/steering3.png}
        			\centering\includegraphics<6>[width=7.2cm]{./resources/steering4.png}
        			\centering\includegraphics<7>[width=7.2cm]{./resources/steering5.png}
        			\centering\includegraphics<8>[width=7.2cm]{./resources/steering6.png}
        			\centering\includegraphics<9>[width=7.2cm]{./resources/steering7.png}
        			\centering\includegraphics<10>[width=7.2cm]{./resources/steering8.png}
        			\centering\includegraphics<11>[width=7.2cm]{./resources/steering9.png}
        			\centering\includegraphics<12>[width=7.2cm]{./resources/steering10.png}
            \end{frame}
            \begin{frame}{Cybernetics}
                \framesubtitle{Approaches}
        		\begin{itemize}
        			\item<1-> \textcolor{teal}{Control}, \textcolor{teal}{feedback}, \textcolor{teal}{communication}, \textcolor{teal}{circular causality} $\longrightarrow$ from N. Wiener and the transdisciplinary endeavours at Macy Conferences (1946-1953):\\
        		    Many many forms -- applications in engineering, management, law, business, ...
        			\item<2-> When applied recursively to examine cybernetic ideas and institutions $\longrightarrow$ \textcolor{teal}{The cybernetics of} \textbf{\textcolor{teal}{observing}} (rather than \textcolor{orange}{observed}) systems: Second-Order Cybernetics\\
        			\textcolor{teal}{Constructive}, \textcolor{teal}{recursive}, and \textcolor{teal}{consistent} approach to cybernetics
        			\item<3-> Important figures and institutions: Kenneth Boulding, Margaret Mead, Heinz von Foerster, Gordon Pask, Humberto Maturana, Paul Pangaro, American Society for Cybernetics, International Society for the Systems Sciences, ...
        		\end{itemize}
            \end{frame}
            \begin{frame}{Cybernetics}
                \framesubtitle{Types of Systems \cite{pangaro_web}}
        		\begin{figure}
        		    \centering
        		    \includegraphics[height=0.75\textheight]{resources/systems.PNG}
        		\end{figure}
            \end{frame}
    \section{Considerations}
        \subsection{Design}
            \begin{frame}{Design}
                From this perspective, design is:
                \begin{itemize}
                    \item<1-> Not the outcome of a process.
                    \item<2-> Not a problem solving process.
                    \item<3-> Not a way of facing complexity.
                    \item<4-> An \textcolor{teal}{activity} that is carried out in the face of very complex and conflicting requirements.
                \end{itemize}
            \end{frame}
        \subsection{First-Order Cybernetics}
            \begin{frame}{Control}
                \begin{itemize}
                    \item<1-> Control $\neq$ Restriction
                    \item<2-> Control $=$ Guide towards better performance\\
                    $\equiv$ Enabling for effectiveness\\
                    $\equiv$ Effective management (Stafford Beer)
                    \begin{itemize}
                        \item Goal or intention
                        \item Means of communication of \textcolor{teal}{the intention} and \textcolor{teal}{the action of control} to an actor
                    \end{itemize}
                    \item<3-> What constitutes control in a system that \textcolor{teal}{enables} rather than \textcolor{red}{restricts}?
                \end{itemize}
            \end{frame}
            \begin{frame}{Law of Requisite Variety (First Law of Cybernetics)}
                \begin{itemize}
                    \item<1-> Variety = the number of all possible states for a system
                    \item<2-> Ross Ashby's law of \textcolor{teal}{Requisite Variety}: \\
                        In order to maintain its viability, the system that is in control must have at least as many states as the system to be controlled.
                    \item<3-> Not being restrictive $\longrightarrow$ The controller must have requisite variety.
                \end{itemize}
            \end{frame}
            \begin{frame}{Back to Control}
                \begin{columns}
                    \column{0.6\textwidth}
                        \begin{itemize}
                            \item<1-> In circular systems, which element controls and which is being controlled?
                            \item<2-> Control is \textcolor{red}{neither} in one element \textcolor{red}{nor} the other; it \textcolor{teal}{emerges} as a result of their \textcolor{teal}{interaction}.
                            \item<3-> Circularity is embodied in the role of the observer in cybernetic systems:\\
                            $\longrightarrow$ The observer cannot be inactive, or there would be no system.
                        \end{itemize}
                    \column{0.4\textwidth}
                    \begin{figure}
        		        \centering\includegraphics[height=0.6\textheight]{resources/thermostate.PNG}
        		    \end{figure}
                \end{columns}
            \end{frame}
        \subsection{Circularity and Second-Order Cybernetics}
            \begin{frame}{Subject/Metasubject}
                \begin{columns}
                    \column{0.5\textwidth}
                        \begin{itemize}
                            \item<1-> Mathematics:\\
                                A Distinct field worthy of consideration in its own right and by its own criteria.\\
                                Its \textcolor{teal}{abstraction} $\longrightarrow$ when applied to other subjects, \textcolor{teal}{casts light} upon them and \textcolor{teal}{enhances} our understanding.
                            \item<2-> Cybernetics $\sim$ Mathematics\\
                                Second-Order Cybernetics $\longrightarrow$ It is both its own subject and its own metasubject.
                            \item<3-> Design $\sim$ Cybernetics\\
                                Studied in the light of its own criteria:\\
                                (recursive) design of design
                        \end{itemize}
                    \column{0.5\textwidth}
                        \begin{itemize}
                            \item<4-> Cybernetics: \textcolor{teal}{structure} and \textcolor{teal}{form} to construct \textcolor{orange}{individual} meaning and understanding\\
                                Leaves \textcolor{teal}{meaning} and \textcolor{teal}{emotion} to the observer (\textcolor{red}{private})\\
                                $\longrightarrow$ Supports structures that support our \textcolor{teal}{autonomy} and \textcolor{teal}{freedom}\\
                                $\longrightarrow$ (recursive) \textcolor{teal}{Support of support}
                        \end{itemize}
                \end{columns}
            \end{frame}
            \subsubsection{Subjective/Objective}
                \begin{frame}
                    \frametitle{Subjective/Objective}
                    \framesubtitle{Observer Incorporated?}
                    \begin{columns}
                        \column{0.7\textwidth}
                            \begin{itemize}
                                \item<1-> Feedback $\longrightarrow$ an active actor (observer)
                                \item<2-> A description (observation) $\implies$ a describer (observer)
                                    \begin{enumerate}
                                        \item Observations are \textcolor{red}{not absolute} but \textcolor{teal}{relative} to the observer's point of view (Einstein)
                                        \item Observations \textcolor{teal}{affect} the observer to obliterate her hope of prediction (Heisenberg)
                                    \end{enumerate}
                                \item<3-> A description (of the world) $\implies$ a universal reality
                                \item<4-> A description (theory) of a describer has to account the \textcolor{teal}{describer} herself and her \textcolor{teal}{expression of the description}.
                            \end{itemize}
                        \column{0.3\textwidth}
                            ``I am a liar'' -- true or false? \\
            				\centering\includegraphics[height=0.5\textheight]{./resources/man.png}\\
                			``The reality is only my imagination'' (?)
                    \end{columns}
                \end{frame}
            \subsubsection{Recursion/Repetition}
                \begin{frame}
                    \frametitle{Recursion/Repetition}
                    \framesubtitle{Heinz von Foerster's Eigen Forms (structures, functions, objects, behaviours, and values)}
                    Tokens for stable behaviour: tokens for Eigen functions (Recursive Function Theory)\\
                    Examples:
                    \begin{itemize}
                        \item<1-> $f(x)= {{x}\over{2}} +1$\\
                            \fontsize{8}{9.2}\selectfont
                            \begin{columns}
                                \column{0.1\textwidth}
                                \column{0.4\textwidth}
                                    $\longrightarrow x_0=4$\\
                                    $\longrightarrow x_1=f(x_0) = {{4}\over{2}} + 1 = 3$ \\
                                    $\longrightarrow x_2=f(x_1) = {{3}\over{2}} + 1 = 2.5$ \\
                                    $\longrightarrow x_3=f(x_2) = {{2.5}\over{2}} + 1 = 2.25$\\
                                    $\longrightarrow x_4=f(x_3) = {{2.25}\over{2}} + 1 = 2.125$\\
                                    $\longrightarrow x_5=f(x_4) = {{2.125}\over{2}} + 1 = 2.063$\\
                                    $\longrightarrow x_{11}=f(x_{10}) = {{x_{10}}\over{2}} + 1 = 2.001$\\
                                    $\longrightarrow x_\infty=f(x_\infty) = {{x_\infty}\over{2}} + 1 =$ \textcolor{red}{$2.000$}\\
                                \column{0.5\textwidth}
                                    $\longrightarrow x_0=1$\\
                                    $\longrightarrow x_1=f(x_0) = {{1}\over{2}} + 1 = 1.5$ \\
                                    $\longrightarrow x_2=f(x_1) = {{1.5}\over{2}} + 1 = 1.75$ \\
                                    $\longrightarrow x_3=f(x_2) = {{1.75}\over{2}} + 1 = 1.875$\\
                                    $\longrightarrow x_8=f(x_7) = {{x_7}\over{2}} + 1 = 1.996$\\
                                    $\longrightarrow x_{10}=f(x_9) = {{x_9}\over{2}} + 1 = 1.999$\\
                                    $\longrightarrow x_\infty=f(x_\infty) = {{x_\infty}\over{2}} + 1 =$ \textcolor{red}{$2.000$}\\ 
                            \end{columns}
                        \item<2-> 2 is the \textcolor{teal}{Eigen value} of $f$
                    \end{itemize}
                \end{frame}
                \begin{frame}
                    \frametitle{Recursion/Repetition}
                    \framesubtitle{Heinz von Foerster's Eigen Forms (structures, functions, objects, behaviours, and values)}
                    Tokens for stable behaviour: tokens for Eigen functions (Recursive Function Theory)\\
                    Examples:
                    \begin{itemize}
                        \item<1-> $f(g)= {{d}\over{dx}}g$\\
                            $exp=e^x \implies f(exp) = {{d}\over{dx}}exp = exp$\\
                            $\longrightarrow f(f(f(...f(f(exp))...))) = exp$\\
                        \item<2-> $exp = e^x$ is the \textcolor{teal}{Eigen function} for operator $f$
                    \end{itemize}
                \end{frame}
                \begin{frame}
                    \frametitle{Recursion/Repetition}
                    \framesubtitle{Heinz von Foerster's Eigen Forms (structures, functions, objects, behaviours, and values)}
                    Tokens for stable behaviour: tokens for Eigen functions (Recursive Function Theory)\\
                    Examples:
                    \begin{itemize}
                        \item<1-> ``This sentence has ... letters'' $\implies$ \textcolor{red}{THIRTY-ONE} or \textcolor{red}{THIRTY-THREE}\\
                        ``This sentence \textcolor{blue}{`This sentence has THIRTY-ONE letters'} has ... letters'' $\implies$ \textcolor{red}{THIRTY-ONE}\\
                        ``This sentence \textcolor{purple}{`This sentence \textcolor{blue}{<This sentence has THIRTY-ONE letters>} has THIRTY-ONE letters'} has ... letters'' $\implies$ \textcolor{red}{THIRTY-ONE} \\
                        \vspace{0.7cm}
                        \item<2-> ``This sentence consists of ... letters'' $\implies$ THIRTY-NINE \\
                        ``This sentence is composed of ... letters'' $\implies \nexists$
                    \end{itemize}
                \end{frame}
                \begin{frame}{Recursion/Repetition}
                    \framesubtitle{Learning, Autopoiesis, Eigen Forms, and Ranulph Glanville's Object Theory}
                    \begin{figure}
                        \centering\includegraphics[width=0.7\textwidth]{resources/recursion.png}
                    \end{figure}
                \end{frame}
                \begin{frame}{Recursion/Repetition}
                    \framesubtitle{Viable Systems Model (Stafford Beer)}
                    \begin{columns}
                        \column{0.5\textwidth} \centering\includegraphics[height=0.8\textheight]{resources/brain_spinalCord.jpg}
                        \column{0.5\textwidth} \centering\includegraphics[height=0.75\textheight]{resources/VSM.png}
                    \end{columns}
                \end{frame}
            \subsubsection{Conversation}
                \begin{frame}{Conversation Model \cite{pangaro_web}}
                    \framesubtitle{Based on Gordon Pask's Conversation Theory (1976)}
                    \centering\includegraphics<1>[page=23, width=0.75\textwidth]{resources/Pangaro-HCII_Seminar-April_2019-distro.pdf}
                    \centering\includegraphics<2>[page=24, width=0.75\textwidth]{resources/Pangaro-HCII_Seminar-April_2019-distro.pdf}
                    \centering\includegraphics<3>[page=25, width=0.75\textwidth]{resources/Pangaro-HCII_Seminar-April_2019-distro.pdf}
                    \centering\includegraphics<4>[page=26, width=0.75\textwidth]{resources/Pangaro-HCII_Seminar-April_2019-distro.pdf}
                    \centering\includegraphics<5>[page=27, width=0.75\textwidth]{resources/Pangaro-HCII_Seminar-April_2019-distro.pdf}
                    \centering\includegraphics<6>[page=28, width=0.75\textwidth]{resources/Pangaro-HCII_Seminar-April_2019-distro.pdf}
                    \centering\includegraphics<7>[page=29, width=0.75\textwidth]{resources/Pangaro-HCII_Seminar-April_2019-distro.pdf}
                    \centering\includegraphics<8>[page=30, width=0.75\textwidth]{resources/Pangaro-HCII_Seminar-April_2019-distro.pdf}
                    \centering\includegraphics<9>[page=31, width=0.75\textwidth]{resources/Pangaro-HCII_Seminar-April_2019-distro.pdf}
                    \centering\includegraphics<10>[page=32, width=0.75\textwidth]{resources/Pangaro-HCII_Seminar-April_2019-distro.pdf}
                    \centering\includegraphics<11>[page=33, width=0.75\textwidth]{resources/Pangaro-HCII_Seminar-April_2019-distro.pdf}
                    \centering\includegraphics<12>[page=34, width=0.75\textwidth]{resources/Pangaro-HCII_Seminar-April_2019-distro.pdf}
                \end{frame}
                \begin{frame}{The Theory}
                    \framesubtitle{Gordon Pask's Conversation Theory (1976)}
                    \centering\includegraphics<1>[page=99, width=0.75\textwidth]{resources/Pangaro-HCII_Seminar-April_2019-distro.pdf}
                \end{frame}
                \begin{frame}{Mental Models \cite{dubberly2009}}
                    \framesubtitle{Abductive Reasoning}
                    \centering \includegraphics[width=\textwidth]{resources/models1.PNG}
                \end{frame}
                \begin{frame}{Mental Models \cite{dubberly2009}}
                    \framesubtitle{Learning as Forming and Reforming Models}
                    \centering\includegraphics[width=0.75\textwidth]{resources/learning.PNG}
                \end{frame}
                \begin{frame}{Mental Models \cite{dubberly2009}}
                    \framesubtitle{TV dramas shape culture or culture inspire TV dramas?}
                    \centering\includegraphics<1->[width=0.75\textwidth]{resources/models4.PNG} \\
                    \centering\includegraphics<2->[width=0.25\textwidth]{resources/chickenegg.jpg}
                \end{frame}
                \begin{frame}{Mental Models \cite{dubberly2009}}
                    \framesubtitle{Agreement}
                    \centering\includegraphics[width=0.5\textwidth]{resources/models5.PNG}
                \end{frame}
    \section{The Harmony}
        \subsection{Determinism}
            \begin{frame}{Cybernetics, Design, and Determinism}
            \framesubtitle{Variety and Design: The Undefinable}
    		    \begin{itemize}
                    \item<1-> The vast universe is nevertheless \textcolor{red}{finite}\\
    		        $\implies$ computing capacity \textcolor{red}{always} has a limit (even the most powerful conceivable systems)
    		        \item<2-> Even simple problems can very rapidly become \textcolor{red}{transcomputable}.\\
    		            Ex: configurations of lightbulbs on a 20$\times$20 matrix
    		            Design almost always faces transcomputable problems.
		            \item<3-> The new: something that is not inherent in the existing \\
		            $\implies$ not predictable $\equiv$ not dependent on any variable $\equiv$ not emergent \\
    		        The Undefinable: Impossible to expect to \textcolor{red}{accurately} define a design problem.
    		    \end{itemize}
            \end{frame}
            \begin{frame}{Cybernetics, Design, and Determinism}
            \framesubtitle{Variety and Design: Definability and Variety}
    		    \begin{itemize}
    		        \item<1-> Viable Control = Effective Management = Regulation = Stability = Sustainability = ... \\
    		        $\implies$ \textcolor{teal}{Requisite Variety} (controll\textcolor{red}{er} $\leqslant$ controll\textcolor{red}{ed})
    		        \item<2-> Second-Order Cybernetics: Control is emergent (distinctiveness of controlled and controller is arbitrary).\\
        		        $\longrightarrow$ For second-order systems: stability $\equiv$ the variety can only be \textcolor{teal}{the same}.
                    \item<3-> Defining a problem = modeling = controlling (without restricting) the performance of a system\\
                    \item<4-> What if the system (the problem under study) is transcomputable?
                        \begin{enumerate}
                            \item<5-> Changing the definitions of the context under which we chose the states (simplifying)
                            \item<6-> Transforming the notion of control: control-as-restricting
                        \end{enumerate}
    		    \end{itemize}
            \end{frame}
            \begin{frame}{Cybernetics, Design, and Determinism}
            \framesubtitle{Variety and Design: Unmanageability and Creativity}
    		    \begin{itemize}
    		        \item<1-> Being in control = Defining the range of possible (what will be considered)
    		        \item<2-> When I am in control $\implies$ I restrict the world to what I can imagine or permit\\
    		            $\implies$ I support a predetermination of what-is and what-might-be and aim towards predetermined goals.
		            \item<3-> Ex: Deciding to go to a restaurant:\\
		                    If I choose (restrict) $\longrightarrow$ reflections of my taste, knowledge and ignorance.\\
		                    If I let others choose $\longrightarrow$ I can find new experiences.
                    \item<4-> Benefit of not having enough variety to control the system\\
                    $\longrightarrow$ Discover \textcolor{teal}{the unexpected} and \textcolor{teal}{the novel}
                    \item<5-> Not restricting the future considerations to current knowledge: Experiencing the vast unknown
    		    \end{itemize}
            \end{frame}
        \subsection{Design As Done}
            \begin{frame}{Design As Done}
            \framesubtitle{Design as a Conversation with self (and others)}
    		    \begin{itemize}
    		        \item<1-> We look and then we draw
    		        \item<1-> We see something new, not previously intended
    		        \item<2-> Conversation requires switching roles: \textcolor{teal}{pendulum}
    		        \item<3-> Meanings are private $\implies$ understandings are different: \textcolor{teal}{novelty}
    		        \item<4-> Agree to agree; or agree to disagree
    		        \item<4-> Test for understanding: Consistency of what I hear and what I say
    		        \item<5-> $\longrightarrow$ Increasing the \textcolor{teal}{variety} of the ``repertoire'' of the designer
    		        \item<6-> Recursion, role switching $\longrightarrow$ understand \textcolor{red}{well enough} to face \textcolor{red}{complexity}
		            \item<7-> We \textcolor{teal}{construct} our understanding
    		    \end{itemize}
            \end{frame}
            \begin{frame}{Design As Done}
            \framesubtitle{Conversation, Objects, Eigen forms, and Autonomy}
    		    \begin{itemize}
    		        \item<1-> Conversation with self
    		        \item<2-> Autopoietic systems: Create conditions to create themselves; continue to \textcolor{teal}{reproduce} themselves in the environment
    		        \item<3-> Eigen forms: a system can operate on its own output, \textcolor{teal}{treating} it as an input.
    		        \item<4-> Designers: create the conditions in which the design outcome can come into being; continue the design act to reach a constant outcome (details  may be enriched).
    		        \item<5-> Children: grow to become their own persons
    		        \item<6-> $\longrightarrow$ \textcolor{red}{Autonomy = Organizational Closure}
    		    \end{itemize}
            \end{frame}
            \begin{frame}{Design As Done}
            \framesubtitle{Conversation as Design}
    		    \begin{itemize}
    		        \item<1-> Design translated once more: \textcolor{red}{Wandering}
                    \item<2-> Once arrived, we can make sense of the progress. (the ``solution'' defines the ``problem'')
                    \item<3-> Arriving and stopping? Eigen forms? \textcolor{teal}{Well enough: stable form}.
                    \item<4-> Stopping and re-starting? 
                    \item<5-> Pendulum: For a second-order description, observers \textcolor{red}{out} must become observers \textcolor{red}{in}.
    		    \end{itemize}
            \end{frame}
            \begin{frame}{Design As Done}
            \framesubtitle{Being Out of Control}
    		    \begin{itemize}
    		        \item<1-> Not being in control $\implies$ a way of increasing creativity
                    \item<2-> Restricting responses and conversation to what I know\\
                        $\implies$ Destroying conversation
                        $\implies$ Conversation controlled by a participant $\longrightarrow \nexists$
                    \item<3-> Design is a second-order activity: Ashby's law cannot be utilised.
                    \item<4-> Stopping and re-starting? 
                    \item<5-> Pendulum: For a second-order description, observers \textcolor{red}{out} must become observers \textcolor{red}{in}.
    		    \end{itemize}
            \end{frame}
            \begin{frame}{Design As Done}
            \framesubtitle{Complexity}
    		    \begin{itemize}
    		        \item<1-> What if the system (the problem under study) is transcomputable?
                        \begin{enumerate}
                            \item Changing the definitions of the context (simplifying)
                            \item Transforming the notion of control: control-as-restricting
                        \end{enumerate}
    		        \item<1-> Designer: someone who faces \textcolor{red}{complexity} (and ambiguity)
                    \item<2-> ``The systems theorist of the future, I suggest, must be an expert in how to simplify'' \\--- Ross Ashby
                    \item<3-> ``Progress means simplifying, not complicating'' --- Bruno Munari (Italian Designer)
                    \item<4-> Simplifying: a process by which complex requirements can be brought together within \textcolor{teal}{one}, \textcolor{teal}{unified}, \textcolor{teal}{unitary} form.
                    \item<4-> Design brings to complexity an approach that is distinct from complexity science, which can lead to added value.
    		    \end{itemize}
            \end{frame}
        \subsection{Criteria and Conditions}
            \begin{frame}{From Cybernetics to Design}
    		    \begin{itemize}
    		        \item<1-> Design leads not to the best, but to a large variety of different outcomes, giving \textcolor{red}{choice}.
    		        \item<2-> The nature of design conversation:\\
    		        For it to operate $\implies \exists$ a listener (viewer)\\
    		        To listen (see) $\implies \exists$ an open (and receptive) mind, and generosity\\
    		        To design $\implies$ see the possibilities \textcolor{red}{not already existent} in the mind
    		        To increase choices through design $\implies$ acting responsibly
    		        \item<3-> Cybernetic systems work if we accept responsibility, and act generously, with an open mind.
    		        \item<4-> Conversation is the intersection
    		    \end{itemize}
            \end{frame}
        \subsection{Notes on Epistemology}
            \begin{frame}{Knowledge Of and Knowledge For}
    		    \begin{itemize}
    		        \item<1-> Science \textcolor{red}{through an observer} gives us \textcolor{red}{knowledge of} the system. \\
    		        Helps us understand \textcolor{red}{what is} $\longrightarrow$ passive and neutral \\
    		        $\implies$ \textcolor{orange}{Tacit} knowledge
    		        \item<2-> Designers are actors \textcolor{red}{in} the world $\longrightarrow$ require and generate \textcolor{red}{knowledge for} acting in the world as they make it.\\
    		        $\implies$ \textcolor{teal}{Reflective} knowledge
    		        \item<3-> Technology (engineering): converts `knowledge of' into `knowledge for'
    		    \end{itemize}
		    \end{frame}
    \section{Discussion}
        \begin{frame}{Wicked Problems}
		    \centering\includegraphics[height=0.8\textheight]{resources/rittle.png}
        \end{frame}
        \begin{frame}{Cybernetics and Design: Conversations for Action \cite{Dubberly2019}}
		    \centering\includegraphics<1>[page=31, width=0.75\textwidth]{resources/Pangaro_Vienna-HvF-Lecture_June-2017_distro-r.pdf}
		    \centering\includegraphics<2>[page=32, width=0.75\textwidth]{resources/Pangaro_Vienna-HvF-Lecture_June-2017_distro-r.pdf}
		    \centering\includegraphics<3>[page=33, width=0.75\textwidth]{resources/Pangaro_Vienna-HvF-Lecture_June-2017_distro-r.pdf}
		    \centering\includegraphics<4>[page=34, width=0.75\textwidth]{resources/Pangaro_Vienna-HvF-Lecture_June-2017_distro-r.pdf}
		    \centering\includegraphics<5>[page=35, width=0.75\textwidth]{resources/Pangaro_Vienna-HvF-Lecture_June-2017_distro-r.pdf}
		    \centering\includegraphics<6>[page=36, width=0.75\textwidth]{resources/Pangaro_Vienna-HvF-Lecture_June-2017_distro-r.pdf}
		    \centering\includegraphics<7>[page=37, width=0.75\textwidth]{resources/Pangaro_Vienna-HvF-Lecture_June-2017_distro-r.pdf}
		    \centering\includegraphics<8>[page=38, width=0.75\textwidth]{resources/Pangaro_Vienna-HvF-Lecture_June-2017_distro-r.pdf}
		    \centering\includegraphics<9>[page=39, width=0.75\textwidth]{resources/Pangaro_Vienna-HvF-Lecture_June-2017_distro-r.pdf}
        \end{frame}
        \begin{frame}{Interaction \cite{pangaro_web}}
    		\begin{figure}
    		    \centering
    		    \includegraphics<1>[height=0.75\textheight]{resources/interaction.PNG}
    		    \centering
    		    \includegraphics<2>[height=0.75\textheight]{resources/conversing1.png}
    		    \centering
    		    \includegraphics<3>[height=0.75\textheight]{resources/conversing2.png}
    		\end{figure}
        \end{frame}
        \begin{frame}{Conversations? \cite{pangaro_web}}
            \centering\includegraphics<1>[page=35, width=0.75\textwidth]{resources/Pangaro-HCII_Seminar-April_2019-distro.pdf}
            \centering\includegraphics<2>[page=36, width=0.75\textwidth]{resources/Pangaro-HCII_Seminar-April_2019-distro.pdf}
            \centering\includegraphics<3>[page=37, width=0.75\textwidth]{resources/Pangaro-HCII_Seminar-April_2019-distro.pdf}
        \end{frame}
        \begin{frame}{\cite{pangaro_web}}
            \centering\includegraphics<1>[page=41, width=0.75\textwidth]{resources/Pangaro_Vienna-HvF-Lecture_June-2017_distro-r.pdf}
            \centering\includegraphics<2>[page=42, width=0.75\textwidth]{resources/Pangaro_Vienna-HvF-Lecture_June-2017_distro-r.pdf}
            \centering\includegraphics<3>[page=44, width=0.75\textwidth]{resources/Pangaro_Vienna-HvF-Lecture_June-2017_distro-r.pdf}
            \centering\includegraphics<4>[page=89, width=0.75\textwidth]{resources/Pangaro_Vienna-HvF-Lecture_June-2017_distro-r.pdf}
            \centering\includegraphics<5>[page=90, width=0.75\textwidth]{resources/Pangaro_Vienna-HvF-Lecture_June-2017_distro-r.pdf}
        \end{frame}
        \begin{frame}{Implications \cite{pangaro_web}}
            \centering\includegraphics<1>[page=92, width=0.75\textwidth]{resources/Pangaro_Vienna-HvF-Lecture_June-2017_distro-r.pdf}
            \centering\includegraphics<2>[page=93, width=0.75\textwidth]{resources/Pangaro_Vienna-HvF-Lecture_June-2017_distro-r.pdf}
            \centering\includegraphics<3>[page=94, width=0.75\textwidth]{resources/Pangaro_Vienna-HvF-Lecture_June-2017_distro-r.pdf}
            \centering\includegraphics<4>[page=95, width=0.75\textwidth]{resources/Pangaro_Vienna-HvF-Lecture_June-2017_distro-r.pdf}
            \centering\includegraphics<5>[page=96, width=0.75\textwidth]{resources/Pangaro_Vienna-HvF-Lecture_June-2017_distro-r.pdf}
            \centering\includegraphics<6>[page=97, width=0.75\textwidth]{resources/Pangaro_Vienna-HvF-Lecture_June-2017_distro-r.pdf}
            \centering\includegraphics<7>[page=98, width=0.75\textwidth]{resources/Pangaro_Vienna-HvF-Lecture_June-2017_distro-r.pdf}
            \centering\includegraphics<8>[page=99, width=0.75\textwidth]{resources/Pangaro_Vienna-HvF-Lecture_June-2017_distro-r.pdf}
            \centering\includegraphics<9>[page=100, width=0.75\textwidth]{resources/Pangaro_Vienna-HvF-Lecture_June-2017_distro-r.pdf}
            \centering\includegraphics<10>[page=101, width=0.75\textwidth]{resources/Pangaro_Vienna-HvF-Lecture_June-2017_distro-r.pdf}
            \centering\includegraphics<11>[page=102, width=0.75\textwidth]{resources/Pangaro_Vienna-HvF-Lecture_June-2017_distro-r.pdf}
            \centering\includegraphics<12>[page=103, width=0.75\textwidth]{resources/Pangaro_Vienna-HvF-Lecture_June-2017_distro-r.pdf}
            \centering\includegraphics<13>[page=104, width=0.75\textwidth]{resources/Pangaro_Vienna-HvF-Lecture_June-2017_distro-r.pdf}
            \centering\includegraphics<14>[page=105, width=0.75\textwidth]{resources/Pangaro_Vienna-HvF-Lecture_June-2017_distro-r.pdf}
            \centering\includegraphics<15>[page=106, width=0.75\textwidth]{resources/Pangaro_Vienna-HvF-Lecture_June-2017_distro-r.pdf}
            \centering\includegraphics<16>[page=107, width=0.75\textwidth]{resources/Pangaro_Vienna-HvF-Lecture_June-2017_distro-r.pdf}
            \centering\includegraphics<17>[page=108, width=0.75\textwidth]{resources/Pangaro_Vienna-HvF-Lecture_June-2017_distro-r.pdf}
            \centering\includegraphics<18>[page=109, width=0.75\textwidth]{resources/Pangaro_Vienna-HvF-Lecture_June-2017_distro-r.pdf}
            \centering\includegraphics<19>[page=110, width=0.75\textwidth]{resources/Pangaro_Vienna-HvF-Lecture_June-2017_distro-r.pdf}
            \centering\includegraphics<20>[page=111, width=0.75\textwidth]{resources/Pangaro_Vienna-HvF-Lecture_June-2017_distro-r.pdf}
            \centering\includegraphics<21>[page=112, width=0.75\textwidth]{resources/Pangaro_Vienna-HvF-Lecture_June-2017_distro-r.pdf}
            \centering\includegraphics<22>[page=113, width=0.75\textwidth]{resources/Pangaro_Vienna-HvF-Lecture_June-2017_distro-r.pdf}
            \centering\includegraphics<23>[page=114, width=0.75\textwidth]{resources/Pangaro_Vienna-HvF-Lecture_June-2017_distro-r.pdf}
            \centering\includegraphics<24>[page=115, width=0.75\textwidth]{resources/Pangaro_Vienna-HvF-Lecture_June-2017_distro-r.pdf}
            \centering\includegraphics<25>[page=127, width=0.75\textwidth]{resources/Pangaro_Vienna-HvF-Lecture_June-2017_distro-r.pdf}
            \centering\includegraphics<26>[page=128, width=0.75\textwidth]{resources/Pangaro_Vienna-HvF-Lecture_June-2017_distro-r.pdf}
        \end{frame}
        \begin{frame}{Design of Ethical Interfaces \cite{pangaro_web}}
            \centering\includegraphics<1>[page=12, width=0.75\textwidth]{resources/Pangaro-HCII_Seminar-April_2019-distro.pdf}
            \centering\includegraphics<2>[page=13, width=0.75\textwidth]{resources/Pangaro-HCII_Seminar-April_2019-distro.pdf}
            \centering\includegraphics<3>[page=14, width=0.75\textwidth]{resources/Pangaro-HCII_Seminar-April_2019-distro.pdf}
            \centering\includegraphics<4>[page=15, width=0.75\textwidth]{resources/Pangaro-HCII_Seminar-April_2019-distro.pdf}
            \centering\includegraphics<5>[page=16, width=0.75\textwidth]{resources/Pangaro-HCII_Seminar-April_2019-distro.pdf}
            \centering\includegraphics<6>[page=17, width=0.75\textwidth]{resources/Pangaro-HCII_Seminar-April_2019-distro.pdf}
            \centering\includegraphics<7>[page=49, width=0.75\textwidth]{resources/Pangaro-HCII_Seminar-April_2019-distro.pdf}
            \centering\includegraphics<8>[page=18, width=0.75\textwidth]{resources/Pangaro-HCII_Seminar-April_2019-distro.pdf}
            \centering\includegraphics<9>[page=19, width=0.75\textwidth]{resources/Pangaro-HCII_Seminar-April_2019-distro.pdf}
            \centering\includegraphics<10>[page=20, width=0.75\textwidth]{resources/Pangaro-HCII_Seminar-April_2019-distro.pdf}
            \centering\includegraphics<11>[page=21, width=0.75\textwidth]{resources/Pangaro-HCII_Seminar-April_2019-distro.pdf}
        \end{frame}
                
    \section*{Overview}
    	\begin{frame}
		\frametitle{Overview}
    		\tableofcontents
    	\end{frame}
        


\section*{References}
	\begin{frame}[allowframebreaks]
		\frametitle{References}
		\printbibliography
	\end{frame}

\section*{Appendix}
	\begin{frame}
		\setbeamercovered{transparent}
		\frametitle{Definition(s) of Cybernetics}
		\framesubtitle{Soooo... What is Cybernetics?}
		\begin{itemize}
			\item<1->Ship of the State: The Command of a naval vessel is a metaphor for the governance of a city/state.\\
			--- Plato (420s-340s B.C.)
			\item<2->``Cybernetique is the art of governing or the science of government.''\\
			--- André-Marie Ampère (1775-1836)
			\item<3->``Use the word ‘cybernetics’, Norbert, because nobody knows what it means. This will always put you at an advantage in arguments.''\\
			Widely quoted; attributed to Claude Shannon in a letter to Norbert Wiener in the 1940s.
		\end{itemize}
		\setbeamercovered{invisible}
	\end{frame}
	\begin{frame}{What is NOT a Conversation?}
        \framesubtitle{Shannon's Model of Transmission}
        \begin{figure}
            \centering\includegraphics[width=0.8\textwidth]{resources/transmission.PNG}
        \end{figure}
        This model does not scaffold novelty, is pure mechanical, and does not support learning.
    \end{frame}
    \begin{frame}{Interaction}
        \framesubtitle{Forming an interaction - Meaning is PRIVATE}
        \centering\includegraphics[height=0.75\textheight]{resources/interaction1.png}
    \end{frame}
    \begin{frame}{Relationship}
        \framesubtitle{Interactions Flowing in Both Direction + Time + Recursion = TRUST}
        \centering\includegraphics[height=0.75\textheight]{resources/interaction2.png}
    \end{frame}
    \begin{frame}{Conversation}
        \framesubtitle{``Immaterial'' Aspects}
        \centering\includegraphics[height=0.75\textheight]{resources/interaction3.png}
    \end{frame}
    \begin{frame}{Mental Models}
        \framesubtitle{Shaping and re-Shaping}
            \centering\includegraphics<1>[width=0.75\textwidth]{resources/models6.PNG}
            \centering\includegraphics<2>[height=0.77\textheight]{resources/models7.PNG}
    \end{frame}
\end{document}